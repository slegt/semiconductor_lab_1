\begin{table*}
	\centering	
		\input{../plots/resistivity_hall_short.tex}
	\caption{Hall effect quantities of p-Si, ZnO, ZTO and CUI thin film samples.}
	\label{tab:hall_results}
\end{table*}

\section{Results and Discussion}
\subsection{Hall Effect Measurements at Room Temperature}
Measurements were performed on p-type silicon (\ce{p-Si}, $d=\qty{500}{\micro\meter}$), 
zinc oxide (\ce{ZnO}, $d=\qty{1}{\micro\meter}$), zinc tin oxide (\ce{ZTO}, 
$d=\qty{1.3}{\micro\meter}$) and copper iodide (\ce{CuI}, $d=\qty{0.339}{\micro\meter}$) 
thin film samples.
Using the Van der Pauw method, the resistivity, Hall carrier concentration and Hall coefficients 
were determined for each sample.

With four distinct sample contact configurations, four measurements were performed 
per sample. 
The resistivity $\rho$ was calculated using \cref{eq:hall_resistivity}, 
the Hall coefficient $R_{\mathrm{H}}$ using \cref{eq:hall_coefficient_van_der_pauw},
the carrier concentration $n$ using \cref{eq:hall_concentration} 
and the Hall mobility $\mu_{\mathrm{H}}$ using \cref{eq:hall_mobility}.
The results are averaged and summarized in \cref{tab:hall_results}, detailed values
can be found in \cref{tab:hall_results_detail}.

An important observation is the variing sign of $R_{\mathrm{H}}$. 
Since the sign of the Hall coefficient is determined by the majority charge carrier
(positive for holes, negative for electrons), this indicates that holes are the majority
charge carrier for \ce{p-Si} and \ce{CuI}. 
Electrons are the majority charge carrier for \ce{ZnO} and \ce{ZTO}. 

\subsection{Temperature Dependent Hall Effect Measurements}
Temperature dependent Hall effect measurements for a bulk \ce{ZnO} sample 
were performed in the Temperature range from \qty{20}{\kelvin} to \qty{500}{\kelvin}.
Both the Hall mobility and Hall carrier concentration were determined for each temperature and are displayed in 
\cref{fig:zno_hall_effect_n,fig:zno_hall_effect_mu} respectively.

The graph of the Hall carrier concentration in \cref{fig:zno_hall_effect_n} shows a linear decrease with increasing inverse temperature. 
This observation can be explained using the low temperature approximation \cref{eq:n_T_approx_low}:
\begin{align}
	&n = \text{const.} \cdot T^{3/2} \exp\left( \frac{-E_{\mathrm{D}}^{b}}{2 \mathrm{k}T} \right) \\
	\ln(&n T^{-3/2}) = \ln(\text{const.}) - \frac{E_{\mathrm{D}}^{b}}{2 \mathrm{k}} \cdot \frac{1}{T} 
	\label{eq:n_T_approx_low}
\end{align}
Using a linear fit, one obtains a slope of \num{\taskTwoSlope}. 
Using \cref{eq:n_T_approx_low}, this slope can be used to determine the donor energy 
$E_{\mathrm{D}}^{b} = \qty{\taskTwoEnergy}{\milli \electronvolt}$.

The Hall mobility in \cref{fig:zno_hall_effect_mu} can be parted into three distinct regions.
For every region, the data follows a linear trend, that can be described using three linear fits.
The first region has a slope of \num{\taskTwoSlopeOne}, the second region has a slope of \num{\taskTwoSlopeTwo} 
and the third region has a slope of \num{\taskTwoSlopeThree}.

In a $\log-\log$ plot, a linear function indicates a power law relationship, whereby the slope of the line
corresponds to the exponent of the power law. 
Since the mobility is determined by scattering processes, that
have a characteristic temperature dependence, different scattering mechanisms can
be identified. 
The first region, where $\mu_\mathrm{H} \propto T^{3 / 2}$, can be explained by ionized impurity scattering.
The second region, where $\mu_\mathrm{H} \propto T^{-1 / 2}$, can be explained by piezoelectric potential scattering.
The third region, where $\mu_\mathrm{H} \propto T^{-3 / 2}$, can be explained by deformation potential scattering.



\begin{figure}
	\centering
	\includegraphics{../plots/task_2_n.pdf}
	\caption{Hall carrier concentration of a bulk \ce{ZnO} sample as a function of temperature.}
	\label{fig:zno_hall_effect_n}
\end{figure}
\begin{figure}
	\centering
	\includegraphics{../plots/task_2_mu.pdf}
	\caption{Hall mobility of a bulk \ce{ZnO} sample as a function of temperature.}
	\label{fig:zno_hall_effect_mu}
\end{figure}

\section{Interpretation of an Unusual Data Set}


\begin{table*}
\centering
\input{../plots/resistivity_hall.tex}
\caption{Hall effect quantities of p-Si, ZnO, ZTO and CUI thin film samples.}
\label{tab:hall_results_detail}
\end{table*}

Hier ist ein neuer Absatz.