\section{Results and Discussion}
\subsection{Hall Effect Measurements at Room Temperature}
Measurements were performed on p-type silicon (\ce{p-Si}, $d=\qty{500}{\micro\meter}$), 
zinc oxide (\ce{ZnO}, $d=\qty{1}{\micro\meter}$), zinc tin oxide (ce{ZTO}, 
$d=\qty{1.3}{\micro\meter}$) and copper iodide (\ce{CuI}, $d=\qty{0.339}{\micro\meter}$) 
thin film samples.
Using the Van der Pauw method, the Hall carrier concentration and Hall coefficients were
determined for each sample.

\num{3e-4}

With four distinct sample contact configurations, four measurements were performed 
leading to calculated Hall coefficients and carrier concentrations, using 
\cref{eq:hall_resistivity} and \cref{eq:hall_concentration}.
The results are summarized in \cref{tab:hall_results}, detailed values can be found in 
\cref{tab:hall_results_detailed}.

P-Si: [0.0001076  0.000107   0.0001102  0.00011069]
ZnO: [0.00334056 0.00334056 0.00334056 0.00334056]
ZTO: [0.00019475 0.00019475 0.00019475 0.0001944 ]
CuI: [7.12577599e-07 7.08148451e-07 7.17837132e-07 7.20711272e-07]
\begin{tabular}{lllll}
	\toprule
	quantity & \ce{p-Si} & \ce{ZnO} & \ce{ZTO} & \ce{CuI} \\
	\midrule
	$\rho_1 (\unit{\ohm \milli \meter})$ & 0.107 & 3.341 & 0.195 & 0.000712 \\
	$\rho_2 (\unit{\ohm \milli \meter})$ & 0.107 & 3.341 & 0.195 & 0.000708 \\
	$\rho_3 (\unit{\ohm \milli \meter})$ & 0.110 & 3.341 & 0.195 & 0.000718 \\
	$\rho_4 (\unit{\ohm \milli \meter})$ & 0.111 & 3.341 & 0.194 & 0.000721 \\
	$\rho_\mathrm{avg} (\unit{\ohm\milli \meter})$
\end{tabular}


\subsection{Temperature Dependent Hall Effect Meassurements}
Temperature dependent Hall effect measurements for a bulk \ce{ZnO} sample 
were performed in the Temperature range from \qty{20}{\kelvin} to \qty{500}{\kelvin}.
Both the Hall mobility and Hall carrier concentration were determined for each temperature and are displayed in 
\cref{fig:zno_hall_effect_n,fig:zno_hall_effect_mu} respectively.

The graph of the Hall carrier concentration in \cref{fig:zno_hall_effect_n} shows a linear decrease with increasing inverse temperature. 
This observation can be explained using the low temperature approximation \cref{eq:n_T_approx_low}:
\begin{align}
	&n = \text{const.} \cdot T^{3/2} \exp\left( \frac{-E_{\mathrm{D}}^{b}}{2 \mathrm{k}T} \right) \\
	\ln(&n T^{-3/2}) = \ln(\text{const.}) - \frac{E_{\mathrm{D}}^{b}}{2 \mathrm{k}} \cdot \frac{1}{T}
\end{align}
Using a linear fit, one obtains a slope of 

\begin{figure}
	\centering
	\includegraphics{../plots/task_2_n.pdf}
	\caption{Hall carrier concentration of a bulk \ce{ZnO} sample as a function of temperature.}
	\label{fig:zno_hall_effect_n}
\end{figure}
\begin{figure}
	\centering
	\includegraphics{../plots/task_2_mu.pdf}
	\caption{Hall mobility of a bulk \ce{ZnO} sample as a function of temperature.}
	\label{fig:zno_hall_effect_mu}
\end{figure}

\section{Interpretation of an Unusual Data Set}


\begin{table*}
\centering
\input{../plots/resistivity_hall.tex}
\caption{Hall effect quantities of p-Si, ZnO, ZTO and CUI thin film samples.}
\label{tab:hall_results_detail}
\end{table*}

Hier ist ein neuer Absatz.