\begin{table*}
	\centering	
		\input{../plots/resistivity_hall_short.tex}
	\caption{Hall effect quantities of \ce{p-Si}, \ce{ZnO}, \ce{ZTO} and \ce{CuI} 
	thin film samples.}
	\label{tab:hall_results}
\end{table*}

\section{Results and Discussion}
\subsection{Hall Effect Measurements at Room Temperature}
Measurements were performed on p-type silicon (\ce{p-Si}, $d=\qty{500}{\micro\meter}$), 
zinc oxide (\ce{ZnO}, $d=\qty{1}{\micro\meter}$), zinc tin oxide (\ce{ZTO}, 
$d=\qty{1.3}{\micro\meter}$) and copper iodide (\ce{CuI}, $d=\qty{0.339}{\micro\meter}$) 
thin film samples.
Using the Van der Pauw method, the resistivity, Hall carrier concentration and Hall 
coefficients were determined for each sample.

With four distinct sample contact configurations, four measurements were performed 
per sample. 
The resistivity $\rho$ was calculated using \cref{eq:hall_resistivity}, 
the Hall coefficient $R_{\mathrm{H}}$ using \cref{eq:hall_coefficient_van_der_pauw},
the carrier concentration $n$ using \cref{eq:hall_concentration} 
and the Hall mobility $\mu_{\mathrm{H}}$ using \cref{eq:hall_mobility}.
The results are averaged and summarized in \cref{tab:hall_results}, detailed values
can be found in \cref{tab:hall_results_detail}.

An important observation is the variing sign of $R_{\mathrm{H}}$. 
Since the sign of the Hall coefficient is determined by the majority charge carrier
(positive for holes, negative for electrons), this indicates that holes are the majority
charge carrier for \ce{p-Si} and \ce{CuI}. 
Electrons are the majority charge carrier for \ce{ZnO} and \ce{ZTO}. 

\subsection{Temperature Dependent Hall Effect Measurements}
Temperature dependent Hall effect measurements for a bulk \ce{ZnO} sample 
were performed in the Temperature range from \qty{20}{\kelvin} to \qty{325}{\kelvin}.
Both the Hall mobility and Hall carrier concentration were determined for each 
temperature and are displayed in 
\cref{fig:zno_hall_effect_n,fig:zno_hall_effect_mu} respectively.

The graph of the Hall carrier concentration in \cref{fig:zno_hall_effect_n} shows a 
linear decrease with increasing inverse temperature. 
This observation can be explained using the low temperature approximation 
\cref{eq:n_T_approx_low}:
\begin{align}
	&n = \text{const.} \cdot T^{3/2} \exp\left( \frac{-E_{\mathrm{D}}^{b}}{2 \mathrm{k}T} \right) \\
	\ln(&n T^{-3/2}) = \ln(\text{const.}) - \frac{E_{\mathrm{D}}^{b}}{2 \mathrm{k}} \cdot \frac{1}{T} 
	\label{eq:n_T_approx_low_linear}
\end{align}
Using a linear fit, one obtains a slope of \num{\taskTwoSlope}. 
Using \cref{eq:n_T_approx_low_linear}, this slope can be used to determine the donor energy 
$E_{\mathrm{D}}^{b} = \qty{\taskTwoEnergy}{\milli \electronvolt}$.
For high temperatures, the carrier concentration saturates, since all donors are ionized.
By applying \cref{eq:n_T_approx_high}, the donor concentration can be estimated to be
$N_{\mathrm{D}} = \qty{\taskTwoND}{\per \meter\cubed}$.

The Hall mobility in \cref{fig:zno_hall_effect_mu} can be parted into three distinct 
regions.
For every region, the data follows a linear trend, that can be described using three 
linear fits.
The first region has a slope of \num{\taskTwoSlopeOne}, the second region has a slope of
\num{\taskTwoSlopeTwo} 
and the third region has a slope of \num{\taskTwoSlopeThree}.

In a $\log-\log$ plot, a linear function indicates a power law relationship, whereby the
slope of the line corresponds to the exponent of the power law. 
Since the mobility is determined by scattering processes, that have a characteristic 
temperature dependence, different scattering mechanisms can be identified. 
The first region, where $\mu_\mathrm{H} \propto T^{3 / 2}$, can be explained by 
ionized impurity scattering.
The second region, where $\mu_\mathrm{H} \propto T^{\taskTwoSlopeTwo} \simeq T^{-1/2}$, 
can be explained by piezoelectric potential scattering.
The third region, where $\mu_\mathrm{H} \propto T^{\taskTwoSlopeThree} \simeq T^{-3 / 2}$, can be 

explained by deformation potential scattering.

\begin{figure}
	\centering
	\includegraphics{../plots/task_2_n.pdf}
	\caption{Hall carrier concentration of a bulk \ce{ZnO} sample as a function of temperature.}
	\label{fig:zno_hall_effect_n}
\end{figure}
\begin{figure}
	\centering
	\includegraphics{../plots/task_2_mu.pdf}
	\caption{Hall mobility of a bulk \ce{ZnO} sample as a function of temperature.}
	\label{fig:zno_hall_effect_mu}
\end{figure}

\subsection{Interpretation of an Unusual Data Set}
Temperature dependent Hall effect measurements of a \ce{ZnO} thin film sample
were performed in the temperature range from \qty{20}{\kelvin} to \qty{325}{\kelvin}.
However, the dataset does not seem to match a single donor model, as the carrier 
concentration, see \cref{zno_hall_n} raises with falling temperature.
The uncorrected data is shown in \cref{zno_hall_n}, it does not show a linear trend. 

This could indicate that the sample is not a single donor system. 
However, \citeauthoryear{look} suggest, that the sample is a single donor system, 
but a two-layer Hall analysis needs to be performed to correct the data. 
Since \ce{ZnO} and sapphire have a large lattice mismatch, a highly dislocated region 
at the interface between thin film and substrate is generated during growth. 
This high-density region of stacking faults significantly affects the carrier 
concentration as well as the mobility as a function of temperature and does not
show a linear trend.

For a two-layer Hall analysis, we index the \ce{ZnO} thin film as '\num{1}' and the 
dislocated region at the interface as index '\num{2}'. 
Using \cref{eq:multilayer_sum_1,eq:multilayer_sum_2} we can find the following identities
\begin{align}
	\sigma_{\square}=e\mu_{\mathrm{H}1}n_{ \square 1}
	+e \mu_{\mathrm{H}2} n_{ \square 2} \\
	R_{\square} \sigma_{\square}^{2}=e \mu_{\mathrm{H} 1}^{2} n_{ \square 1} 
	+ e\mu_{\mathrm{H} 2}^{2} n_{ \square 2}
\end{align}
It is possible to find an analytical expression for 
$\mu_\mathrm{H}$ and $n$:
\begin{align}
	\mu_{\mathrm{H}}&=R_{\square} \sigma_{\square}
	=\frac{R_{\square} \sigma_{\square}^{2} /d}{\sigma_{\square} / d} \\
	&=\frac{\mu_{\mathrm{H}1}^{2}n_{1}
	+\mu_\mathrm{H2}^{2} n_{\square2} /d}{\mu_{\mathrm{H}1}n_{1}
	+\mu_{\mathrm{H}2}n_{\square{2}} /d} \\
	n_{}=\frac{n_{ \square}}{d}&=\frac{1}{eR_{\square}d}
	= \frac{\sigma^{2}_{\square}/ d^{2}}{eR_{\square}\sigma_{\square}^{2} / d}  \\
	&=\frac{(\mu_{\mathrm{H}1}n_{1}
	+\mu_{\mathrm{H}2}n_{\square2} /d)^{2}}{\mu_{\mathrm{H}1}^{2}n_{1}
	+\mu_{\mathrm{H}2}^{2} n_{ \square 2} /d}
\end{align}
Since the donor atoms freeze at low temperatures, the interface layer must be dominant 
in this region. 
Since the paper suggest that the layer's carrier concentration and mobility is 
temperature independent, we can approximate
$n_{}=n_{ \square 2} / d = \qty{\taskThreeInterfaceN}{\per\meter\cubed}$ 
and 
$\mu_{\mathrm{H}}=\mu_{\mathrm{H 2}} = \qty{\taskThreeInterfaceM}{\centi\meter\squared\per\volt\per\second}$
for $T \to \qty{0}{\kelvin}$. 

An expression for the corrected hall mobility $\mu_{\mathrm{H} 1}$ and  
hall carrier concentration $n_{1}$ can now 
be found:
\begin{align}
	\mu_{\mathrm{H} 1}=\frac{\mu_{\mathrm{H}}^{2} n_{}- \mu_{2} ^{2} 
	n_{ \square 2} /d}{\mu_{\mathrm{H}} n_{} - \mu_{2} n_{\square 2} / d} \\
	n_{1} = \frac{(\mu_{\mathrm{H}}n_{}-\mu_{2}n_{\square 2} 
	/ d)^{2}}{\mu_{\mathrm{H}}^{2}n_{}-\mu_{2}^{2} n_{\square 2} / d}	
\end{align}
The corrected carrier density is also displayed in \cref{zno_hall_n} and can
now be identified with the single-donor case.
Using linear regression, a slope of \num{\taskThreeSlope} and a corresponding donor 
energy of $\qty{\taskThreeEnergy}{\milli \electronvolt}$ can be determined.
The donor concentration can be estimated using \cref{eq:n_T_approx_high} to be
$N_{\mathrm{D}} = \qty{\taskThreeND}{\per\meter}$

The corrected mobility is displayed in \cref{zno_hall_mu} and shows a linear trend with a 
slope of \num{\taskThreeMobilityOne} for the first region and a slope of 
\num{\taskThreeMobilityThree} for the second region.
The log-log plot indicates a power law relationship, that can be explained by
deformation potential scattering since 
$\mu_\mathrm{H} \propto T^{\taskThreeMobilityThree} \simeq T^{-3/2}$.

\begin{figure}
	\centering
	\includegraphics{../plots/task_3_n.pdf}
	\caption{Corrected and uncorrected Hall carrier concentration of a \ce{ZnO} thin 
	film sample as a function of temperature.}
	\label{zno_hall_n}
\end{figure}

\begin{figure}
	\centering
	\includegraphics{../plots/task_3_mu.pdf}
	\caption{Corrected and uncorrected Hall mobility of a \ce{ZnO} thin film sample as 
	a function of temperature.}
	\label{zno_hall_mu}
\end{figure}

\begin{table*}
\centering
\input{../plots/resistivity_hall.tex}
\caption{Hall effect quantities of p-Si, ZnO, ZTO and CUI thin film samples.}
\label{tab:hall_results_detail}
\end{table*}
