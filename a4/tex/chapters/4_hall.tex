\section{Hall effect}
\begin{figure*}
	\centering
	\includegraphics[width=0.8\linewidth]{../assets/hall_geometry.png}
	\caption{Schematic structure for Hall effect measurements.
		\imcite{grundmann}}
	\label{fig:hall}
\end{figure*}
The geometrical arrangement required for Hall effect measurements
is displayed in \cref{fig:hall}.
A magnetic field $\mathbf{B}=B \cdot \hat{z}$ penetrates
a semiconductor sample.
A current $\mathbf{j}$ is flowing along the $x$-direction inside the sample 
as a response to an external electric field $\mathbf{E}=E_x \cdot \hat{x}$.
Due to the magnetic force, the electrons are deflected and a current in
$y$-direction establishes until the resulting electric force in $y$-direction
fully compensates the magnetic force.
The system equilibrates for $j_y=0$.

To describe the system in detail for one majority charge carrier, one can use the 
equation of motion from relaxation time approximation:
\begin{equation}
	m^{*} \frac{\mathbf{v}}{\tau}=q(\mathbf{E}
	+\mathbf{v}\times \mathbf{B}).
\end{equation}
In this equation, $m^{*}$ is the effective mass of the charge carrier, $\mathbf{v}$ the
average electron velocity and $\tau$ the relaxation time constant.
With $\mathbf{j}=nq\mathbf{v}$ and the resistivity tensor $\hat{\rho}$, 
this vector equation can be brought into a linear transformation of the form
$\mathbf{E}=\hat{\rho}\mathbf{j}$ or more explicit:
\begin{equation}
	\label{eq:mej}
	\begin{pmatrix}
		E_{x} \\
		E_{y} \\
		E_{z}
	\end{pmatrix}
	=
	\begin{pmatrix}
		\frac{m^{*}}{\tau nq^{2}} & - \frac{B_{z}}{nq}        & 0                         \\
		\frac{B_{z}}{nq}          & \frac{m^{*}}{\tau nq^{2}} & 0                         \\
		0                         & 0                         & \frac{m^{*}}{\tau nq^{2}}
	\end{pmatrix}
	\begin{pmatrix}
		j_{x} \\
		j_{y} \\
		j_{z}
	\end{pmatrix}.
\end{equation}
The system equilibrates for $\mathbf{j} = j_x \cdot \hat{x}$ and with \cref{eq:mej}, the following
relationship holds true: $E_{y} = B_{z} / (nq) j_{x} = R_{\mathrm{H}}B_{z}j_{x}$.
This leads to an expression for the carrier density $n$ using directly measurable quantities.
\begin{align}
	R_{\mathrm{H}}&=\frac{1}{nq}=\frac{E_{y}}{j_{x}B_{z}}\\
	n&=\frac{j_{x}B_{z}}{E_{y}q}
\end{align}

Now we want to find an expression for the mobility.
\begin{equation}
	\begin{pmatrix}
		E_{x} \\
		E_{y} \\
		E_{z}
	\end{pmatrix}
	=\begin{pmatrix}
		\mu_\mathrm{H}^{-1} & -B_{z}              & 0                   \\
		B_{z}               & \mu_\mathrm{H}^{-1} & 0                   \\
		0                   & 0                   & \mu_\mathrm{H}^{-1}
	\end{pmatrix}
	\begin{pmatrix}
		v_{x} \\
		v_{y} \\
		v_{z}
	\end{pmatrix}
\end{equation}
The hall mobility in this equation is defined as $\mu_{\mathrm{H}}=\frac{q\tau}{m^{*}}$.
To see why this is useful, first, think of the limit $B_{z}\to 0$.
Then, electric field and velocity indeed follow the relation $\mathbf{v}=\mu_{\mathrm{h}}\mathbf{E}$.
Second with $\mathbf{v}=\mu_{\mathrm{h}}\mathbf{E}$ one can easily see that
$v_{x} = \mu_{\mathrm{H}}E_{x}$ and $E_{y}=B_{z}v_{x}=B_{z}\mu_{\mathrm{H}}E_{x}$.
Therefore we find the following equation for $\mu_\mathrm{H}$:
\begin{equation}
	\mu_{\mathrm{H}}=\frac{E_{y}}{B_{z}E_{x}}
\end{equation}
