\section{Measurement Methods}

\subsection{General Structure}
\begin{figure}[H]
	\centering
	\includegraphics[width=0.95\linewidth]{../assets/aufbau.png}
	\caption{Setup of a scanning electron microscope. \imcite{rem_script}}
	\label{fig:general_structure}
\end{figure}
The schematical structure of a scanning electron microscope is shown in
\cref{fig:general_structure}. It consists of a cathode which emits
electrons. 
The electrons are accelerated by an anode and focused into a 
nanometer-wide beam by multiple magnetic lenses.

The goal of a \ac{sem} is to probe the sample at different spots.
To achive this, one can use the magnetic field of a coil, which deflects
the electron beam. 
With that, the electron beam is rasterizing over the surface. 
Different detectors are located close to the sample to measure different 
interaction-products of the colliding electrons with the sample atoms. 
As a result, it is possible to gain an intensity signal, which
depends on the electron beam position.


For older devices, the rasterized electron beam is synchronized
with a CRT. 
In newer devices, the signal is digitized and can be outputted in
various formats. 

\subsection{Electron Beam Generation}
Electrons can be generated through thermionic emission.
In this experiment, a tungsten hairpin cathode is used.
The cathode is heated up to \qtyrange{2600}{3000}{\kelvin} to overcome
the work function of \qty{2.5}{\electronvolt} through thermal 
excitation.
The cathode is surrounded by a Wehnelt-cylinder, on which a voltage is 
applied.
The cylinder is used as a first focusing mechanism.
This is schematically visualized in \cref{fig:wolfram}. 
\begin{figure}[H]
	\centering
	\includegraphics[width=0.95\linewidth]{../assets/wolfram.png}
	\caption{Geometric arrangement of the cathode and the Wehnelt-cylinder.
	\imcite{rem_script}}
	\label{fig:wolfram}
\end{figure}
There also exist \ce{LaB6} cathodes, which consist of a small rod-shaped
lanthanum hexaboride single crystal. 
This crystal is heated up indirectly to \qtyrange{1700}{2100}{\kelvin}.
Due to the lower work function of \qty{2.7}{\electronvolt} the cathode
can work at a lower temperature. 

Apart from thermionic emission there also exist field emission, where
the quantum mechanical tunneling effect is utilized to generate electrons.
This provides a more precise beam but is also more complex to operate.

