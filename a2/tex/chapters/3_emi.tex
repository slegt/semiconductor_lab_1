\section{Electron Matter Interaction}
To understand the working principle of a \ac{sem}, it is necessary to consider the
interaction between electrons and solid matter.
For that, one observes the path of an electron that is emitted
directly from a cathode
and head to a sample. Those electrons are called \ac{pe}.
After arriving at the surface, the electrons undergo elastic and inelastic scattering.
Due to complex interactions between \ac{pe} and the atoms
inside the sample, multiple interaction products are generated that can
be classified into different categories.
The categories that are relevant for conducting the experiment are the
following:
\begin{itemize}
	\item Secondary electrons
	\item Backscattered electrons
	\item X-ray emission
\end{itemize}
After arriving at the surface, the electrons spread due to
small-angle scattering in a pear-shaped area around the collision point
into the sample.
Different interactions are located at different spatial regions.
This is visualized in \cref{fig:birne}.
The higher the atomic number, the stronger is the scattering of the
electrons.

\subsection{Secondary Electrons}
\ac{pe} collide with bound sample electrons and ionize the
corresponding atoms.
These now free electrons from the top layer can diffuse out of the
material and are known as \ac{se}.
Due to the high energy loss during ionization, \ac{se}
have a low kinetic energy compared to \ac{pe}.
The energy distribution is shown in \cref{fig:electrons}.
To quantify the relation between \ac{pe} and \ac{se}, one
uses the electron yield $\delta_\mathrm{SE} = \text{\# SE} /
	\text{\# PE}$, where $\#$ denotes the number of the respective electrons.
In the typical case of \ac{pe} with energies between
\qtyrange{10}{25}{\kilo\electronvolt}, $\delta_\text{SE}$ is far
below one.

It is possible to get a topographic contrast from \ac{pe}.
The electron yield strongly depends on the angle of incidence of
the surface with $\delta_\mathrm{SE} \simeq \cos(\theta)$.
If there is a change in height, there must also be a change of the
incidence angle $\theta$ which alters the image.
Note that only height changes and not the absolute height define the
topography contrast.
Since the absorption length for secondary electrons is only a couple nanometers thin,
this method delivers precise information
about the local structure.

\begin{figure}[h]
	\centering
	\includegraphics[width=0.95\linewidth]{../assets/birne.png}
	\caption{Interaction area of electrons. \imcite{rem_script}}
	\label{fig:birne}
\end{figure}
\ac{se} are caught by the detector and converted into
an electrical signal.
The scanning electron microscope uses a light-sensitive photomultiplier
with a scintillator disc and a metal grid for collecting electrons.
The metal grid can be biased to filter electrons with different energies.
For \ac{se}, the metal grid is positively charged to attract electrons
from a larger spatial region.
After the electrons collide with the scintillator, the material emits
light, which can be detected by the photomultiplier.
The more electrons collide with the scintillator, the higher is the
intensity of the emitted light and the stronger is the
electrical signal that leads to a brighter pixel in the image.
Not only can the surface topography influence the electron yield, but
the electrical surface potentials can as well.

\begin{figure}
	\centering
	\includegraphics[width=0.95\linewidth]{../assets/elektronen.png}
	\caption{Electron yield as a function of energy. \imcite{rem_script}}
	\label{fig:electrons}
\end{figure}

\subsection{Backscattered Electrons}
A large proportion of \ac{pe} won't ionize the material but
rather be reflected or backscattered by the nuclei of the sample.
Due to the heavy nucleus, the electrons will primarily change their
direction.
These electrons are called \ac{be}. In a physical sense, \ac{be} can't be
distinguished from \ac{se}, but their kinetic energy is much higher
which provides a good indicator.
This leads to the definition that electrons with energies in the order
of magnitude of $\mathrm{e} V_0$, where $\mathrm{e}$ is the elementary
charge and $V_0$ the potential difference of the cathode, are
categorized as \ac{be}.

The electron yield for backscattered electrons depends strongly on
the atomic number $Z$ with $\delta_\text{BS} = \# BS / \# PE
	\propto \sqrt{Z}$.
Because of this dependency and the fact that \ac{be}
contrast is less affected by surface layers and local surface fields,
they offer a reliable
method to detect material contrast.
This is visualized in \cref{fig:material_contrast}.

To detect \ac{be}, one can use a solid-state p-n
junctions with multiple sectors.
Those detectors are reverse biased to establish an electric field that
can collect and count the arriving electrons.
A stronger signal corresponds to a higher number of collected electrons.

\subsection{X-ray Emission}
The X-ray spectrum consists of two sections.
The first part is bremsstrahlung, which is generated during the
deceleration of electrons.
This type of radiation is, due to its origin, present in every X-ray
emission experiment and cannot be used for material identification.

For that purpose, one can use characteristic radiation that makes up
the second part of the X-ray spectrum.
During ionization of the sample atoms, electrons from energetically
higher states relax into energetically more favorable states.
The resulting energy difference generates an X-ray photon.
Due to the discrete energies of the electron states,
the resulting energy-intensity spectra will contain sharp, well-defined
peaks that are characteristic for every material.
The energy transitions $K_{\mathrm{\alpha}_{1}}$ and
$K_{\mathrm{\alpha}_{2}}$ are primarily observed.

To analyze X-ray radiation, one can conduct an \ac{edx} experiment.
A cooled silicon diode detector is driven reverse biased to create an
electric field.
If an X-ray photon enters the material, it will create electron-hole-pairs
which can be separated and counted.
The higher the number of electrons, the higher the energy of the X-ray
photon.
By analyzing the positions of the peaks in the spectrum, qualitative
conclusions about the composition can be drawn.
Through the corresponding intensity, quantitative conclusions are also
possible.

\begin{figure}[h]
	\centering
	\includegraphics[width=0.95\linewidth]{../assets/material.png}
	\caption{Material contrast (left) and topographical contrast (right) detection.
		\imcite{rem_script}}
	\label{fig:material_contrast}
\end{figure}
